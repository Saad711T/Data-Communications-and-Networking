\documentclass[12pt]{article}

\usepackage{amsmath}    % For mathematical symbols and equations
\usepackage{amssymb}    % For additional math symbols
\usepackage{geometry}   % To control page margins
\usepackage{hyperref}   % To include hyperlinks
\usepackage{fancyhdr}   % To customize the header and footer
\usepackage{enumitem}   % For better lists
\usepackage{setspace}   % For line spacing
\usepackage{parskip}    % For better paragraph spacing

\geometry{a4paper, margin=1in}

\pagestyle{fancy}
\fancyhf{}
\fancyhead[L]{Data communications} 
\fancyfoot[C]{\thepage}

\title{Communication System}
\author{0xSaad / Saad Almalki}
\date{\today}

\begin{document}

\maketitle

\section{Model of a Communication System}
\includegraphics[width=0.5\textwidth]{images/model.png} \\

\begin{itemize}
    \item \textbf{1-Information Source} : Message produced by a source may not electrical in nature such as voice , Input tranduced converts message to electrical time varying quantity called message signals , Output transducer converts electricak waveform to appropriate form at destination.
    \item \textbf{2-Transmitter} : Data generated by source can not transmitted directly , Signal processing operation performed by transmitter.
    \item \textbf{3-The communication channel} : Provides electrical connection between distant source and destination , Noise is always random and has great effect on the signals , The channel may be wired or wireless.
    \item \textbf{4-Receiver} : The receiver extracts the input signal from the degraded version coming from the channel.
\end{itemize}

\section{Classification of Communication Systems}

\begin{itemize}
    \item \textbf{1-Analog Systems} : Designed to transmit analog data using analog modulation , Can be wired or wireless , TV-AM-FM are examples.
    \item \textbf{2-Digital Systems} : Designed to transmit digital data using digital or analog modulation , data maybe binary or binary coded of analog data , It terms as Data communications systems.
\end{itemize}

\section{Criteria of Communication Systems}

\begin{itemize}
    \item \textbf{1-Data Delivery} : Data should be delivered to correct destination. 
    \item \textbf{2-Data Integrity} : Data should be delivered without no errors , no addition , no loss.
    \item \textbf{3-Timeliness of data transfer} : Data should be delivered without violating the delay
constraints specific for each service.   
\end{itemize}

\section{Elements of Digital communication System}
\includegraphics[width=0.5\textwidth]{images/model2.png} \\

1- Information source
Based on its output, it may be analog or digital.

2- Source encoder/decoder
- It converts input to a binary sequence of 0‟s and 1‟s.
- The source decoder converts this back to a symbol sequence.

3- Channel encoder/decoder
- Channel encoder adds extra bits to output of the source encoder to detect or correct
errors at the channel decoder.
- Channel encoder/decoder can realize high transmission

4- Modulator/demodulator
\textbf{Modulator} : accepts a bit stream and converts it to an electrical waveform suitable for
transmission over channel.

\textbf{Demodulator} : extracts the message from the information

\section{Why modulation ?}

\begin{itemize}

    \item \textbf{1- Overcoming some equipment limitations}
    \item \textbf{2- Removing interference}
    \item \textbf{3- Reducing noise}
    \item \textbf{4- Allowing efficient capacity utilization}
    \item \textbf{5- Matching signal to channel}
\end{itemize}


\begin{flushright}
Created by: Saad Almalki \\
\end{flushright}

\end{document}