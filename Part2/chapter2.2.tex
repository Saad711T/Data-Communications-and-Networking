\documentclass[12pt]{article}

\usepackage{amsmath}    % For mathematical symbols and equations
\usepackage{amssymb}    % For additional math symbols
\usepackage{geometry}   % To control page margins
\usepackage{hyperref}   % To include hyperlinks
\usepackage{fancyhdr}   % To customize the header and footer
\usepackage{enumitem}   % For better lists
\usepackage{setspace}   % For line spacing
\usepackage{parskip}    % For better paragraph spacing
\usepackage[utf8]{inputenc}
\usepackage{graphicx}
\usepackage{hyperref}
\usepackage{titlesec}
\usepackage{setspace}
\usepackage{lipsum}


\geometry{a4paper, margin=1in}

\pagestyle{fancy}
\fancyhf{}
\fancyhead[L]{Data communications} 
\fancyfoot[C]{\thepage}

\title{Communication System - Part2}
\author{0xSaad / Saad Almalki}
\date{\today}

\begin{document}

\maketitle

\section{Line Configuration}

\begin{itemize}
    \item \textbf{1-Point to Point}
    \item \textbf{2-Multipoint}
\end{itemize}

\subsection{Point to Point}
\includegraphics[width=0.5\textwidth]{images/point-to-point.png} \\
dedicated link between two devices , Entire link capacity is reserved for transmission between two devices ,  It utilized in star and ring topology of computer networks.

\subsection{Multipoint}
\includegraphics[width=0.5\textwidth]{images/multipoint.png} \\
also called multidrop , More than two specific devices share a single link , The capacity of the channel is shared between multiple devices , This type of connection is employed in the bus network topology.

\section{Directions of Data Flow}
\includegraphics[width=0.5\textwidth]{images/dataflows.png} \\

\begin{itemize}
    \item \textbf{1-Simplex} : The transmission is unidirectional - Only one station can transmit; the other can only receive - The paging systems, TV, and FM radio are examples.
    \item \textbf{2-Half Duplex} : Each station can both transmit and receive, but not at the same time - When one station is sending, the other can only receive - The “Push-to-talk” walkie-talkies is half-duplex system.
    \item \textbf{3-Full Duplex} : Both stations can transmit and receive at once - Signals going in either direction share the capacity of the link - One example is telephone network.
\end{itemize}

\section{Transmission Modes}

\begin{itemize}
    \item \textbf{1-Parallel}
    \item \textbf{2-Serial}
\end{itemize}

\subsection{Parallel Transmission}
\includegraphics[width=0.5\textwidth]{images/parallel.png} \\
Multiple bits are sent with each clock tick.

Advantages of Parallel transmission

\begin{itemize}
    \item \textbf{It is characterized by high speed of data transmission.}
    \item \textbf{It can increase the transfer speed by a factor of n over serial transmission.}
\end{itemize}

Disadvantages of Parallel transmission

\begin{itemize}
    \item \textbf{High cost as it requires n lines just to transmit data stream.}
    \item \textbf{It is practical only for short distances.}
    \item \textbf{Consequently, parallel transmission is usually limited to shorter distances}
\end{itemize}


\subsection{Serial Transmission}
\includegraphics[width=0.5\textwidth]{images/serial.png} \\
One bit follows another, so it needs only one channel.

Advantages of Serial transmission

\begin{itemize}
    \item \textbf{Only one channel is required; consequently, cost is reduced.}
    \item \textbf{It has few errors and is practical for long distances.}
\end{itemize}

Disadvantages of Serial transmission

\begin{itemize}
    \item \textbf{It is slow.}
    \item \textbf{There is a need for serial to parallel conversions.}
    \item \textbf{Consequently, serial transmission is usually practical for long distances.}
\end{itemize}


\begin{flushright}
Created by: Saad Almalki \\
\end{flushright}

\end{document}