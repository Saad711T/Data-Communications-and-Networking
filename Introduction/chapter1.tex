\documentclass[12pt]{article}

\usepackage{amsmath}    % For mathematical symbols and equations
\usepackage{amssymb}    % For additional math symbols
\usepackage{geometry}   % To control page margins
\usepackage{hyperref}   % To include hyperlinks
\usepackage{fancyhdr}   % To customize the header and footer
\usepackage{enumitem}   % For better lists
\usepackage{setspace}   % For line spacing
\usepackage{parskip}    % For better paragraph spacing

\geometry{a4paper, margin=1in}

\pagestyle{fancy}
\fancyhf{}
\fancyhead[L]{Data communications} 
\fancyfoot[C]{\thepage}

\title{Introduction to Data Communications}
\author{0xSaad / Saad Almalki}
\date{\today}

\begin{document}

\maketitle

\section{What is Data ?}
Data refers to \textbf{raw facts}

\section{What is Information ?}
Information is \textbf{proccessed data enables us to take decisions.}

\section{What is Data Communications ?}
Data Communications is the \textbf{Process of communicating information in binary form between two parts.}

Also called Digital Communications.
And Computer Communications.

Because most of information intercharged between computers.

\section{Data Types}

\begin{itemize}
    \item \textbf{Text : } Sequence of symbols represented in specific language , converted to binary form from 0s and 1s.
    \item \textbf{Number : } Represented using binary form , not represented in ASCII code but converted to binary form.
    \item \textbf{Image : } Made of pixels
    \item \textbf{Audio : } Refers to sound or music , Continuous not discrete.
    \item \textbf{Video : } Representations of images (called frames) in time.
\end{itemize}


\begin{flushright}
Created by: Saad Almalki \\
\end{flushright}

\end{document}